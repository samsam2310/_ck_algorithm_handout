%----------------------------------------------------------------------------------------
%   PACKAGES AND OTHER DOCUMENT CONFIGURATIONS
%----------------------------------------------------------------------------------------

\documentclass{article}

\usepackage{fancyhdr} % Required for custom headers
\usepackage{lastpage} % Required to determine the last page for the footer
\usepackage{extramarks} % Required for headers and footers
\usepackage[usenames,dvipsnames]{color} % Required for custom colors
\usepackage{graphicx} % Required to insert images
\usepackage{listings} % Required for insertion of code
\usepackage{caption}
\usepackage{courier} % Required for the courier font
\usepackage{lipsum} % Used for inserting dummy 'Lorem ipsum' text into the template
\usepackage{titlesec}

% My addon
\renewcommand{\thepage}{\Roman{page}}% Roman numerals for page counter
\usepackage{fontspec}   %加這個就可以設定字體
\usepackage{xeCJK}       %讓中英文字體分開設置
\setmainfont{Arial}
\setCJKmainfont{思源黑體} %設定中文為系統上的字型,而英文不去更動,使用原TeX字型
\XeTeXlinebreaklocale "zh"             %這兩行一定要加,中文才能自動換行
\XeTeXlinebreakskip = 0pt plus 1pt     %這兩行一定要加,中文才能自動換行

% \def\footnotesize{\fontsize{16}{24}\selectfont}
\def\small{\fontsize{10}{15}\selectfont}
\def\normalsize{\fontsize{12}{16}\selectfont}
\def\large{\fontsize{16}{24}\selectfont}
\def\Large{\fontsize{20}{30}\selectfont}
\def\LARGE{\fontsize{24}{36}\selectfont}
\def\huge{\fontsize{32}{48}\selectfont}
\def\Huge{\fontsize{36}{54}\selectfont}

% Margins
\topmargin=-0.45in
\evensidemargin=0in
\oddsidemargin=0in
\textwidth=6.5in
\textheight=9.0in
\headsep=0.25in

\linespread{1.1} % Line spacing

% Set up the header and footer
\pagestyle{fancy}
\lhead{\hmwkTitle} % Top left header
% \chead{\hmwkClass\ (\hmwkClassInstructor\ \hmwkClassTime): \hmwkTitle} % Top center head
\rhead{\hmwkClass} % Top right header
% \lfoot{\lastxmark} % Bottom left footer
\cfoot{\thepage} % Bottom center footer
% \rfoot{Page\ \ of\ \protect\pageref{LastPage}} % Bottom right footer
\renewcommand\headrulewidth{0.4pt} % Size of the header rule
% \renewcommand\footrulewidth{0.4pt} % Size of the footer rule

\setlength\parindent{0pt} % Removes all indentation from paragraphs

%----------------------------------------------------------------------------------------
%   CODE INCLUSION CONFIGURATION
%----------------------------------------------------------------------------------------

\newcounter{nalg} % defines algorithm counter for chapter-level
\DeclareCaptionLabelFormat{algocaption}{\normalsize\bf Algorithm \thenalg} % defines a new caption label as Algorithm x.y

\lstnewenvironment{algorithm}[1][] %defines the algorithm listing environment
{   
    \refstepcounter{nalg} %increments algorithm number
    \captionsetup{labelformat=algocaption,labelsep=colon} %defines the caption setup for: it ises label format as the declared caption label above and makes label and caption text to be separated by a ':'
    \lstset{ %this is the stype
        frame=tB,
        numbers=left, 
        numberstyle=\normalsize,
        basicstyle=\normalsize, 
        keywordstyle=\color{black}\bfseries\em,
        keywords={,input, output, return, datatype, function, in, if, else, foreach, while, begin, end,} %add the keywords you want, or load a language as Rubens explains in his comment above.
        numbers=left,
        xleftmargin=.04\textwidth,
        #1 % this is to add specific settings to an usage of this environment (for instnce, the caption and referable label)
    }
}
{}


%----------------------------------------------------------------------------------------
%       NAME AND CLASS SECTION
%----------------------------------------------------------------------------------------

% 這裡記得改
\newcommand{\hmwkTitle}{培訓-4} % Assignment title
\newcommand{\hmwkDueDate}{2015年10月8日(木曜日)} % Due date
\newcommand{\hmwkClass}{圖論 I} % Course/class
\newcommand{\hmwkAuthorName}{samsam2310} % Your name

% 不需要修改
\newcommand{\hmwkClassTime}{不需要修改} % Class/lecture time
\newcommand{\hmwkClassInstructor}{不需要修改} % Teacher/lecturer

%----------------------------------------------------------------------------------------
%       TITLE PAGE
%----------------------------------------------------------------------------------------

\title{\hmwkClass}
\author{\hmwkAuthorName}
\date{\hmwkDueDate}

%----------------------------------------------------------------------------------------

\begin{document}
\LARGE~\\[4ex]
\centerline{\bf\hmwkClass}\large\\[2ex]\centerline{\hmwkAuthorName}\\[2ex]\centerline{\hmwkDueDate}\\
\normalsize


%----------------------------------------------------------------------------------------
% 從這裡開始寫
\section{圖論簡介}
\subsection*{Introduction}
圖論是離傘數學的分支,是一門以圖(Graph)為研究對象的理論。
這裡的圖(Graph)跟圖片(Picture)是完全沒有關係的。\\
圖(Graph)是由點(Vertex)和邊(Edge)組成,一張圖就是一群點的集合,而邊是這群點的關係。
圖考慮的是點和點之間的關係,和點的大小、邊的粗細、兩條邊的夾角等等是沒有關係的。
\subsection{名詞解析}
\begin{description}
\item[ 1.]圖(Graph):圖是由點的集合和邊的集合組成的,可以表示為$G = (V,E)$,
其中G是圖,V是點的集合,E是邊的集合。
\item[ 2.]頂點(節點、Vertex、Node):點,組成圖的元素集合,圖G的點集合用$V(G)$表示,點的數量稱為階(order)。
\item[ 3.]邊(Edge):點之間的關係,圖G的邊集合用$E(G)$表示。
用$e(v_1,v_2)$代表一條連接$v_1$和$v_2$的邊,若關係是雙向的,則稱作無向邊(Undirected edge, Edge),
即$e(v_1,v_2) = e(v_2,v_1)$;
若關係是單向的則稱為有向邊(Directed edge),或是弧(Arc),此時$e(v_1,v_2)$代表一條連接$v_1$和$v_2$的單向邊。\\
沒有特別限制的情況下,$v_1$可以等於$v_2$,$v_1$到$v_2$也不一定只有一條邊。
\item[ 4.]相鄰(Adjacent):表達點之間的關係,$v_1$和$v_2$相鄰若且為若存在$e(v_1,v_2)$或$e(v_2,v_1)$。
\item[ 5.]無向圖(Undirected Graph):所有的邊皆為無向邊的圖。
\item[ 6.]有向圖(Directed Graph):所有的邊皆為有向邊的圖。
\item[ 7.]混和圖(Mixed Graph):有有向邊也有無向邊的圖。
\item[ 8.]路徑(Path):點邊交錯的序列,$v_1,e_1,v_2,...,e_n-1,v_n$,可以當作從$v_1$走到$v_n$的一條路,其中所有的
$v_i \in V(G), e_i \in E(G), e_i = e(v_i, v_{i+1})$。
\item[ 9.]簡單路徑(Simple Path):點跟邊都不重複的一條路徑。
\item[ 10.]環(Cycle):一條路徑,至少有一條邊,而且$v_1 = v_n$。
\item[ 11.]簡單環(Simple Cycle):除了起點(終點)之外點邊都不重複的環。
\item[ 12.]度數(Degree):一個點$v$的度數代表該點點連接著多少邊,記做$deg(v)$。
\item[ 13.]入度(In-Degree):從其他點指向點$v$的有向邊數量,記做$deg^+(v)$。
\item[ 14.]出度(Out-Degree):從點$v$指向其他點的有向邊數量,記做$deg^-(v)$。
\end{description}
可以思考一下,$\sum_{v \in V(G)}deg(v) = 2|E|$,
$\sum_{v \in V(G)}deg^+(v) = \sum_{v \in V(G)}deg^-(v) = |E|$(握手定理)。

\subsection{存圖的方法}
存圖的方法會直接影響圖論演算法的複雜度,為了解決資訊競賽中的圖論問題,我們必須要思考怎麼樣有效率的存一張圖,以下有幾個方法。

\begin{description}
\item[ 1.]Adjacent Matrix 鄰接矩陣:以一個矩陣A紀錄一張圖G。
$a_{ij}$紀錄邊$e(v_i,v_j)$,的資訊(是否存在或是權重多少)。
優點是實作非常簡單,可以$O(1)$查詢修改任意邊的資訊,
缺點是無法處理重邊,空間和時間複雜度最少都是$O(|V|^2)$
\item[ 2.]Adjacent List 鄰接列表:對每一個點開一個列表(可以是任何的資料結構,比如vector、link list)紀錄該點和與他相鄰的邊的資訊,
優點是空間複雜度降到了$O(|E|+|V|)$、可以記錄重邊,
缺點是實作稍微複雜一點,插入或刪除邊的方式取決於所使用的資料結構。
\item[ 3.]Forward Star 前向星:將所有的邊存在一個陣列裡面,並依照某種順序排序,使得每一個點$v$連出去的邊都在一個區間裡,通常對於邊$(v_i,v_j)$,先照$v_i$大小排序,如果相同再照$v_j$大小排序。
是一個很好的資料結構,大部分操作都是線性的,只是實作麻煩。
\end{description}


\section{子圖與特殊圖}
\subsection{名詞解析}
\begin{description}
\item[ 1.]子圖 Subgraph:如果兩個圖$G = (V,E), G' = (V',E')$且$V' \subseteq V, E' \subseteq E$,則$G'$為$G$的子圖。
\item[ 2.]補圖 Compilment Graph:圖$G$的補圖$G^c$,滿足$V(G^c) = V(G)$且$e \in E(G) \leftrightarrow e \notin E(G^c)$。
\item[ 3.]生成樹 Spanning tree:
\end{description}


% 範例題目的格式
\subsection{Exercises!!}
\begin{description}
\item[ 1.]<TIOJ 1080 逆序數對>\\
輸入一個數列,輸出數列中有多少數對是逆序的。
\item[ 2.]<TIOJ 1355 河內之塔-蘿莉塔>\\
有三根柱子,第一根上有圈圈,每個大小都不一樣,且大的在下面,然後請輸出如何把所有的圈圈移到第三根柱子上,過程中不能讓小的圈圈在大的圈圈下面。
\item[ 3.]<TIOJ 1500 平面最近點對>\\
給你平面上N個點,問最近的兩個點有多近。
\item[ 4.]<POJ 1741 樹分治>\\
給你一棵樹,問樹上距離小於等於K的點對數量。
\item[ 5.]<TIOJ 1631 點連接遊戲>\\
在正方形xy平面上給你N個紅點跟綠點(左上右上紅點,左下右下綠點),保證三點不共線,請用不交叉的數個線段分別把紅點跟綠點連起來。
\end{description}



% caption 是標題,裡面是虛擬碼。
\begin{algorithm}[caption={Binary Search}, label={alg1}]
function Binary_Search(x, array[])
    l = 0
    r = n - 1 (n是array的大小)
    while l < r
        mid = (l + r) / 2
        if x > array[mid]
            l = mid + 1
        else
            r = mid
    return l
end function
\end{algorithm}


\end{document}