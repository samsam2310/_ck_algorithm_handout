%----------------------------------------------------------------------------------------
%   PACKAGES AND OTHER DOCUMENT CONFIGURATIONS
%----------------------------------------------------------------------------------------

\documentclass{article}

\usepackage{fancyhdr} % Required for custom headers
\usepackage{lastpage} % Required to determine the last page for the footer
\usepackage{extramarks} % Required for headers and footers
\usepackage[usenames,dvipsnames]{color} % Required for custom colors
\usepackage{graphicx} % Required to insert images
\usepackage{listings} % Required for insertion of code
\usepackage{caption}
\usepackage{courier} % Required for the courier font
\usepackage{lipsum} % Used for inserting dummy 'Lorem ipsum' text into the template
\usepackage{titlesec}

% My addon
\renewcommand{\thepage}{\Roman{page}}% Roman numerals for page counter
\usepackage{fontspec}   %加這個就可以設定字體
\usepackage{xeCJK}       %讓中英文字體分開設置
\setmainfont{Arial}
\setCJKmainfont{思源黑體} %設定中文為系統上的字型,而英文不去更動,使用原TeX字型
\XeTeXlinebreaklocale "zh"             %這兩行一定要加,中文才能自動換行
\XeTeXlinebreakskip = 0pt plus 1pt     %這兩行一定要加,中文才能自動換行

% \def\footnotesize{\fontsize{16}{24}\selectfont}
\def\small{\fontsize{10}{15}\selectfont}
\def\normalsize{\fontsize{12}{16}\selectfont}
\def\large{\fontsize{16}{24}\selectfont}
\def\Large{\fontsize{20}{30}\selectfont}
\def\LARGE{\fontsize{24}{36}\selectfont}
\def\huge{\fontsize{32}{48}\selectfont}
\def\Huge{\fontsize{36}{54}\selectfont}

% Margins
\topmargin=-0.45in
\evensidemargin=0in
\oddsidemargin=0in
\textwidth=6.5in
\textheight=9.0in
\headsep=0.25in

\linespread{1.1} % Line spacing

% Set up the header and footer
\pagestyle{fancy}
\lhead{\hmwkTitle} % Top left header
% \chead{\hmwkClass\ (\hmwkClassInstructor\ \hmwkClassTime): \hmwkTitle} % Top center head
\rhead{\hmwkClass} % Top right header
% \lfoot{\lastxmark} % Bottom left footer
\cfoot{\thepage} % Bottom center footer
% \rfoot{Page\ \ of\ \protect\pageref{LastPage}} % Bottom right footer
\renewcommand\headrulewidth{0.4pt} % Size of the header rule
% \renewcommand\footrulewidth{0.4pt} % Size of the footer rule

\setlength\parindent{0pt} % Removes all indentation from paragraphs

%----------------------------------------------------------------------------------------
%   CODE INCLUSION CONFIGURATION
%----------------------------------------------------------------------------------------

\newcounter{nalg} % defines algorithm counter for chapter-level
\DeclareCaptionLabelFormat{algocaption}{\normalsize\bf Algorithm \thenalg} % defines a new caption label as Algorithm x.y

\lstnewenvironment{algorithm}[1][] %defines the algorithm listing environment
{   
    \refstepcounter{nalg} %increments algorithm number
    \captionsetup{labelformat=algocaption,labelsep=colon} %defines the caption setup for: it ises label format as the declared caption label above and makes label and caption text to be separated by a ':'
    \lstset{ %this is the stype
        frame=tB,
        numbers=left, 
        numberstyle=\normalsize,
        basicstyle=\normalsize, 
        keywordstyle=\color{black}\bfseries\em,
        keywords={,input, output, return, datatype, function, in, if, else, foreach, while, begin, end,} %add the keywords you want, or load a language as Rubens explains in his comment above.
        numbers=left,
        xleftmargin=.04\textwidth,
        #1 % this is to add specific settings to an usage of this environment (for instnce, the caption and referable label)
    }
}
{}


%----------------------------------------------------------------------------------------
%       NAME AND CLASS SECTION
%----------------------------------------------------------------------------------------

% 這裡記得改
\newcommand{\hmwkTitle}{培訓-5} % Assignment title
\newcommand{\hmwkDueDate}{2015年10月20日(火曜日)} % Due date
\newcommand{\hmwkClass}{RMQ問題と塊狀資料結構} % Course/class
\newcommand{\hmwkAuthorName}{samsam2310} % Your name

% 不需要修改
\newcommand{\hmwkClassTime}{不需要修改} % Class/lecture time
\newcommand{\hmwkClassInstructor}{不需要修改} % Teacher/lecturer

%----------------------------------------------------------------------------------------
%       TITLE PAGE
%----------------------------------------------------------------------------------------

\title{\hmwkClass}
\author{\hmwkAuthorName}
\date{\hmwkDueDate}

%----------------------------------------------------------------------------------------

\begin{document}
\LARGE~\\[4ex]
\centerline{\bf\hmwkClass}\large\\[2ex]\centerline{\hmwkAuthorName}\\[2ex]\centerline{\hmwkDueDate}\\
\normalsize


%----------------------------------------------------------------------------------------
% 從這裡開始寫
\section{RMQ問題與線段樹}
\subsection*{Introduction}
這裡說的線段樹,其實是一個競賽領域產稱的一個新的資料結構,以前還有另外一種線段樹,
但是在這種線段樹出現之後,網路上幾乎都稱這種資料結構為線段樹。\\
線段樹基本上是一顆完整二元搜尋樹,可以用來處理RMQ問題(在競賽時又常常被稱為10萬區間操作題)。\\
所謂的RMQ,就是Dynamic Range Minimum Query,\\
給定一個序列 S[1, n], 以及以下兩種操作:\\
\begin{description}
\item[ 1.]將 S[i] 改為 k
\item[ 2.]查詢 S[i, j] 中的最小值
\end{description}
以上就是一個很經典的RMQ問題,這種問題通常會要求一個區間經過計算處理之後的資訊,
並同時要求隨時更新其中的點。而處理這種問題有多種資料結構,我們一一介紹。


\subsection{離線處理}
RMQ問題通常N都是10萬,也就是我們必須做到$O(N lg N)$的複雜度。
我們不難發現,如果完全不更新,我們其實是可以利用動態規劃等等方法預處理,
讓複雜度降到$O(N)$處理,$O(1)$查詢,但是如果每次更新我們就必須重新預處理,
複雜度退化到$O(N^2)$;而我們也可以反向操作,$O(N)$查詢,$O(1)$更新,
但是在查詢量多時一樣會退化到$O(N^2)$。\\
這時我們想要做的就是均攤更新和查詢的處理時間,這樣就不會因為一邊的操作數太多而超時。
我們可以先將所有操作吃進來,根據操作數量決定要使用上述兩種方法中的哪一種,這種方法稱為離線。
通常在問題優化很複雜、高階資料結構難以實作時,這是一個可以輕鬆獲得大量分數的方法,
甚至有時搭配排序之後的動態規劃,甚至可以拿到所有分數。

\subsection{塊狀練表}
其實我們處理的問題,每次都是詢問一個區間,所以其實我們不一定要更新整塊。
我們可以先把所有區間切成$\sqrt N$塊,每次更新,就把該點所在的那塊全部更新,
查詢時,就$O(\sqrt N)$跑過每一塊計算答案。
這樣做的複雜度變成更新和查詢都是$O(\sqrt N)$,而10萬的$O(\sqrt N)$和$O(lg N)$沒有差非常多,
複雜度可以接受。

\subsection{線段樹}
承接塊狀練表的概念,那如果我在$\sqrt N$塊區間上再覆蓋一些區間呢?
我們發現,我們可以不斷細分區間的寬度來達到均攤更新與查詢的目的,這時線段樹就出現了。\\
線段樹,是一個完美二元樹(平衡而且優先填滿左節點),每一個節點都代表一個區間。
第一層代表全部的區間,第二層開始每層都是上一層的對半切,遞迴建構到區間寬度是1。
概念有點類似分治,當我們切到底之後可以輕鬆處理答案,之後每一層的值都由左右子節點合併求出。\\
很明顯的,線段樹的深度是$lg N$,每次更新只要處理其中$lg N$個節點,
查詢查詢則是接近$2 lg N$個節點(左界跟右界),所以複雜度就是$O(N lg N)$。



\section{實作一顆線段樹}
要實作一顆基本的線段樹,我們要實作更新和查詢兩種方法。\\
要存一顆線段樹可以用陣列或是用指標,陣列型線段樹比較單純,而指標型線段樹則比較好擴展成持久化線段樹等等的資料結構。\\

\subsection{建構}
陣列式線段樹,因為線段樹是完美二元樹,所以我們可以用Array[1]當作根節點,
Array[2*n]、Array[2*n+1]當作第n個點的左、右子節點。\\
至於建構時,我們可以用遞迴函數來建構。
傳入現在的節點編號,區間的左界跟右界,然後分別遞迴到左右子節點,並且將區間對切。

\subsection{更新}
更新有點類似建構,只是更簡單,我們只要處理有變化的點就好。

\subsection{查詢}
查詢,因為查詢的區間不可能剛好等於我們的區間,所以我們必須湊出來。
遞迴查詢時,我們可以分出三種狀況,屬於區間、不屬於區間、剛好切到區間的邊邊。
對於這三種狀況,如果是第一種我們可以直接計算它的值、第二種可以直接忽略、第三種則需要繼續遞迴處理。


\section{區間更新與懶惰標記}
當更新的點不是一個而是一個區間時怎麼辦呢?線段樹也可以處理。
作法是在每個節點上加上一個懶惰標記(lazy tag),這個標記可以是boolean也可以是有意義的數字。
當我們區間更新時,我們跟查詢一樣先分成三種區間,包含、不包含、部分包含,對於後面兩種,分別是忽略和繼續遞迴處理,而第一種,我們發現底下的區間都必須更新,那我們就先記錄一個懶惰標記,當之後遇到再繼續處理。\\
當我們在進行任何操作時,如果要遞迴的時候發現這個區間有懶惰標記,必須要先處理懶惰標記才可以繼續遞迴,這樣才能保證線段樹的正確性。


\section{線段樹的變形}
\subsection{Binary Indexed Tree}
BIT是線段樹的一種變形,實做簡單,但是功能較少,通常是用來處理動態前綴和。
他可以處理:
\begin{description}
\item[ 1.]單一修改, 區間 (0, k] 查詢
\item[ 2.]區間 (0, k] 修改, 單一查詢
\item[ 3.]可藉由排容原理所得到的值 (如總和)
\end{description}
比較有趣的是BIT的實作,因為我們現在只關心從頭開始的區間,所以我們可以將線段樹的右節點全部刪去,
形成一顆只有左節點的樹,而此時我們就發現,這些左節點的右界都不一樣,可以透過這些節點得到全部的前綴。
歸納其記錄的區間我們得到一個公式,索引值為 k 的節點,其所維護的區間是 (k − 2i, k],i 稱為 k 的lowbit,即為二進位表示中從最低位數來第一個非 0 的位數,lowbit 可以由以下的方法算出:
$$2^l = k\&(−k)$$
查詢 (0, k] 區間時,其結果便是 (k − lowbit(k), k] 和 (0, k − lowbit(k)] 合併後的結果。前者恰
好就是 BIT 在 k 所維護的區間,而後者只需遞迴下去計算即可。對於單一操作,則可以發現 $k_0 = k, k_j = k_j−1 + lowbit(k_j−1)$ 恰好就是覆蓋 k 的所有區間。由於只須維持 (0, k]的區間,故兩者的複雜度皆為 $O(lg n)$。

\subsection{K 維線段樹}
對於一個 2 維的線段樹,其每個節點是一棵樹,第一層樹紀錄 x 坐標,而此 x 樹的節點是一個紀錄 y 坐標的樹,而 k 維則依此類推。不過我們通常不會用超過 2 維 (甚至連 2維都很少用) 這是由於空間複雜度的限制。2 維線段樹有個致命傷,便是無法進行區塊修改 + 區塊詢問,因為在二維以上的區間,會出現兩區塊在不同維坐標互相包含的情況,此情形會造成同一個點被許多不相為父子節點的區間包含,因此無法實作。線段樹有另一個generalization 叫做 Quadtree 和 Octree,它們操作的時間複雜度會比較高一些。\\
至於樹狀數組的高維度則比較容易實作,以二維為例只須建立一個二維陣列 S,其中 $S[i][j]$維護 $(i − lowbit(i), i] × (j − lowbit(j), j]$ 區塊之值,其操作和一維的類似,只是複雜度變為$O(lg^2 n)$。


\end{document}