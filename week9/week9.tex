%----------------------------------------------------------------------------------------
%   PACKAGES AND OTHER DOCUMENT CONFIGURATIONS
%----------------------------------------------------------------------------------------

\documentclass{article}

\usepackage{fancyhdr} % Required for custom headers
\usepackage{lastpage} % Required to determine the last page for the footer
\usepackage{extramarks} % Required for headers and footers
\usepackage[usenames,dvipsnames]{color} % Required for custom colors
\usepackage{graphicx} % Required to insert images
\usepackage{listings} % Required for insertion of code
\usepackage{caption}
\usepackage{courier} % Required for the courier font
\usepackage{lipsum} % Used for inserting dummy 'Lorem ipsum' text into the template
\usepackage{titlesec}

% My addon
\renewcommand{\thepage}{\Roman{page}}% Roman numerals for page counter
\usepackage{fontspec}   %加這個就可以設定字體
\usepackage{xeCJK}       %讓中英文字體分開設置
\setmainfont{Arial}
\setCJKmainfont{思源黑體} %設定中文為系統上的字型,而英文不去更動,使用原TeX字型
\setCJKmainfont{WenQuanYi Micro Hei} %
\XeTeXlinebreaklocale "zh"             %這兩行一定要加,中文才能自動換行
\XeTeXlinebreakskip = 0pt plus 1pt     %這兩行一定要加,中文才能自動換行

% \def\footnotesize{\fontsize{16}{24}\selectfont}
\def\small{\fontsize{10}{15}\selectfont}
\def\normalsize{\fontsize{12}{16}\selectfont}
\def\large{\fontsize{16}{24}\selectfont}
\def\Large{\fontsize{20}{30}\selectfont}
\def\LARGE{\fontsize{24}{36}\selectfont}
\def\huge{\fontsize{32}{48}\selectfont}
\def\Huge{\fontsize{36}{54}\selectfont}

% Margins
\topmargin=-0.45in
\evensidemargin=0in
\oddsidemargin=0in
\textwidth=6.5in
\textheight=9.0in
\headsep=0.25in

\linespread{1.1} % Line spacing

% Set up the header and footer
\pagestyle{fancy}
\lhead{\hmwkTitle} % Top left header
% \chead{\hmwkClass\ (\hmwkClassInstructor\ \hmwkClassTime): \hmwkTitle} % Top center head
\rhead{\hmwkClass} % Top right header
% \lfoot{\lastxmark} % Bottom left footer
\cfoot{\thepage} % Bottom center footer
% \rfoot{Page\ \ of\ \protect\pageref{LastPage}} % Bottom right footer
\renewcommand\headrulewidth{0.4pt} % Size of the header rule
% \renewcommand\footrulewidth{0.4pt} % Size of the footer rule

\setlength\parindent{0pt} % Removes all indentation from paragraphs

%----------------------------------------------------------------------------------------
%   CODE INCLUSION CONFIGURATION
%----------------------------------------------------------------------------------------

\newcounter{nalg} % defines algorithm counter for chapter-level
\DeclareCaptionLabelFormat{algocaption}{\normalsize\bf Algorithm \thenalg} % defines a new caption label as Algorithm x.y

\lstnewenvironment{algorithm}[1][] %defines the algorithm listing environment
{   
    \refstepcounter{nalg} %increments algorithm number
    \captionsetup{labelformat=algocaption,labelsep=colon} %defines the caption setup for: it ises label format as the declared caption label above and makes label and caption text to be separated by a ':'
    \lstset{ %this is the stype
        frame=tB,
        numbers=left, 
        numberstyle=\normalsize,
        basicstyle=\normalsize, 
        keywordstyle=\color{black}\bfseries\em,
        keywords={,input, output, return, datatype, function, in, if, else, foreach, while, begin, end,} %add the keywords you want, or load a language as Rubens explains in his comment above.
        numbers=left,
        xleftmargin=.04\textwidth,
        #1 % this is to add specific settings to an usage of this environment (for instnce, the caption and referable label)
    }
}
{}


%----------------------------------------------------------------------------------------
%       NAME AND CLASS SECTION
%----------------------------------------------------------------------------------------

% 這裡記得改
\newcommand{\hmwkTitle}{培訓-1} % Assignment title
\newcommand{\hmwkDueDate}{2015年11月18日(水曜日)} % Due date
\newcommand{\hmwkClass}{DP2と特殊解題技巧} % Course/class
\newcommand{\hmwkAuthorName}{samsam2310} % Your name

% 不需要修改
\newcommand{\hmwkClassTime}{不需要修改} % Class/lecture time
\newcommand{\hmwkClassInstructor}{不需要修改} % Teacher/lecturer

%----------------------------------------------------------------------------------------
%       TITLE PAGE
%----------------------------------------------------------------------------------------

\title{\hmwkClass}
\author{\hmwkAuthorName}
\date{\hmwkDueDate}

%----------------------------------------------------------------------------------------

\begin{document}
\LARGE~\\[4ex]
\centerline{\bf\hmwkClass}\large\\[2ex]\centerline{\hmwkAuthorName}\\[2ex]\centerline{\hmwkDueDate}\\
\normalsize


%----------------------------------------------------------------------------------------
% 從這裡開始寫
% 標題那些的
\section{DP優化}

在前面的章節我們提到了用動態規劃 (Dynamic Programming) 來解決各式各樣的問題,
而在這裡我們要討論一些神奇的優化方式。

\subsection{滾動}
這算是特別提一下,通常在DP得時候,記憶體都會很大,有時會超過我們的允許範圍(比如10的8次方有時時間可以接受,但是空間不可以),這時我們可以觀察一下我們的轉移式,然後利用某個維度只會利用到前幾層的性質,讓兩排記憶體交錯使用,減少記憶體的空間。(可以透過複製一遍或是交換指標或是用變數取值)

\subsection{單調隊列優化}
先回憶一下雙向佇列 (Deque) 這個資料結構,他可以支援兩種操作:\\
• 從兩端 push 新的元素進去。\\
• 從兩端 pop 元素出去。\\
可以知道 Deque 完全涵蓋了 Stack 和 Queue 的操作。\\
而 Deque 通常用來優化 DP 中取極值的動作,尤其是要查詢多個範圍的極值,且這些範圍滿
足一定的單調性,我們將在下面一一討論。
\subsubsection{$Ans_i = min_{L_i ≤ j ≤ R_i}v_j ; L_i ≤ L_i +1, R_i ≤ R_i +1 $}
簡單來說我們要依序查詢序列的某些區間的最大值,而且這些序列的區間左界和右界都
是遞增的。假設序列的長度為 N, 如果我們對於每個區間都直接掃過一遍,時間複雜度會到
$O(N^2)$。
但仔細觀察可以發現,如果存在兩個元素 vk, vh 使得 vk ≥ vh 且 k < h ≤ Ri,那麼對於第 i
筆之後的詢問,vk 絕不可能是區間的最小值。因此我們可以維護 Deque 從前面到後面是遞
增的,即加入元素時保持單調性而從後面 pop 出元素,而當要查詢的時候先從前端 pop 出過
期的元素後,極值即為 Deque 中最前端的元素。
\subsubsection{$Ans_i = min_{L_i ≤ j ≤ i} vj + (di − dj ) ; L_i ≤ L_i +1$}
對於這個情況,如果我們令 ai = vi − di,原本的式子就可以改寫成
$$ Ans_i = min_{L_i ≤ j ≤ i} a_j + d_i$$
而因為對於同一個 i 來說 di 是固定的,這樣就轉化成第一種情況了。這個式子可以讓我們加
速背包問題的 DP,回憶一下背包問題的 DP 式,假設一個物品的重量為 w,價值為 v 且有
c 個:
$$ D(n, m) = max_{0≤k≤c;0≤m−wk} D(n − 1, m − wk) + vk$$
這個式子可以改寫為
$$ D(n, wi + r) = max_{0≤j≤i} D(n − 1, wj + r) + v(i − j)$$
可以看出 D(n − 1, wj + r) 即為 vj,vi 即為 di。這可以讓我們在 O(W N) 的時間解決背包問
題,其中 W 是背包的耐重,N 為物品的數量,是一個與同一個物品的數量 c 無關的算法。
\subsubsection{$Ans_i = max_{Li≤j≤i} d_j t_i + v_j; L_i ≤ L_i +1, d_i ≤ d_i +1, t_i ≤ t_i +1$}
這個情況即是所謂的斜率優化,對於一個 j,我們可以把每一個 dj ti + vj 看做是的一條
直線 y = djx + vj,這些線的斜率都是遞增的,而在我們加入一條條的直線,維持凸包的單
調性,也就是說我們在加入新的一條線 l 時,假設 Deque 中最後兩條線為 l1, l2,如果 l1 超
過 l2 的 x 座標比 l 超過 l2 的 x 座標大,那我們直接將 l1 pop 出,而每次查詢從前面 pop 出
過期或是已經被後面的線超過的元素。

\subsection{四邊形優化}
四邊形優化是一個非常恐怖的東西,當你對 DP 的熱愛程度爆表了可以考慮研究一下。
首先我們先定義兩種 DP 問題。
\subsubsection{1D/1D DP}
這種式子的標準式
$$Di = min_{i≤k<i} Dj + w(j, i)$$
簡單來說就是 Di 會它前面得出的答案 Dj 再加上一個轉移的函數 w(i, j) 中的最小值。
\subsubsection{1D/2D DP}
標準式
$$Di,j = min_{0≤k<i} D_i,k + D_k +1,j + w(i, j) ; D_i,i = 0 $$
這種 DP 式的意義大概是一個區間 [i, j] 的答案會是將這個區間分成兩塊 [i, k], [k + 1, j] 的分
法中的最小值,再加上一個與 k 無關的權。
\subsubsection{四邊形單調性}
我們將單調性分成兩種。
• 凹四邊形單調性
如果對於任何 a < b, c < d 且 F(a, c) ≤ F(b, c),就有 F(a, d) ≤ F(b, d)。
• 凸四邊形單調性
如果對於任何 a < b, c < d 且 F(a, c) ≥ F(b, c),就有 F(a, d) ≥ F(b, d)。
但直接驗證 F 滿足這個性質往往是不容易的,因此我們常用以下不等式檢驗。
• 凹四邊形不等式
如果對於任何 a < b, c < d 都有 F(a, c) + F(b, d) ≥ F(a, d) + F(b, c),則 F 滿足凹四邊
形單調性。
• 凸四邊形不等式
如果對於任何 a < b, c < d 都有 F(a, c) + F(b, d) ≤ F(a, d) + F(b, c),則 F 滿足凸四邊
形單調性。
更進一步的結果,我們只需要檢查 F(i, j) + F(i + 1, j + 1) 和 F(i + 1, j) + F(i, j + 1) 的大小
關係即可。
\subsubsection{1D/1D 凹性優化}
回憶一下 1D/1D 的 DP 式:
$$Di = min_{i≤k<i} D_j + w(i, j)$$
如果我們令 F(j, i) = Dj + w(j, i),代表用 j 轉移 i 的花費,而 Di 就相當於 F(i, k) 中的最小值。\\
此時可以知道如果 w(j, i) 滿足凹四邊形單調性,F(j, i) 也會滿足凹四邊形單調性。代表對於
i, i + 1, j, j + 1 來說,一旦 F(j, i) ≤ F(j + 1, i),也就是說如果用 j 來轉移 i 較 j + 1 來轉移i 好的話,那麼必有 F(j, i + 1) ≤ F(j + 1, i + 1),即接下來用 j 轉移 i + 1 也會較用 j + 1 來的好!換句話說如果我們用 ki 代表用哪一個 j 值來轉移 i 會得到最小的值,必有 ki ≥ ki+1。\\
利用這個單調性,我們便可以用特殊的方法加速。我們使用 Stack 維護當前最佳解,對於
stack 裡的元素 s 我們紀錄 (L, R, p),代表對於所有 L ≤ i ≤ R,我們用 j = p 來轉移到 i 最佳。因此當我們要求 Di 時只需先從 Stack pop 出過期的元素 (即 R < i),之後把 Stack 的
top 元素計算 F(p, i) 即可。而當我們要將 j = i 加入 Stack 時,假設 Stack 的 top 元素為 s,有幾種情況:\\
1. F(i, s.L) ≥ F(s.p, s.L)\\
這代表用 p 轉移完勝 i,因此我們保留 s 在 Stack 中。\\
2. F(i, s.R) ≤ F(s.p, s.R)\\
與上面相反,這代表用 i 轉移完勝 p,因此我們直接將 s 給 pop 出。\\
3. F(i, s.L) ≤ F(s.p, s.L), F(i, s.R) ≥ F(s.p, s.R)\\
這時候代表存在一個界線,使得在界線前 i 較 p 好,而在後面 p 較 i 好。由單調性我們可以使用二分搜找出這個界線 L′,並將 s 更新為 (L′, R, p)。\\
最後我們再將 i 加入 Stack 中,即加入 (i + 1, s.L − 1, i),便完成更新了。(當然有些時候我
們根本無需加入,如 s.L − 1 < i + 1 時。複雜度為 O(N lg N)。\\
\subsubsection{1D/1D 凸性優化}
與凹性相反,我們用 ki 代表用哪一個 j 值來轉移 i 會得到最小的值,此時則為 ki ≤ ki+1。
這時候我們也用和剛才類似的方法優化,只不過我們將 Stack 改為 Deque,取值的時候先從
前端 pop 出過期的元素,而加入新的值時,我們就從後邊刪除元素,最後一樣二分搜出確切
位置。
\subsubsection{2D/1D 凹性優化}
2D/1D 凹性優化並沒有特別特殊的性質,我們可以考慮枚舉 i 後每次視為 1D/1D 的問題。
令 Fi(j′) = Di,i+j′,此時有
$$Fi(j′) = min_{0≤k<j′} Fi(k) + ui(k, j′)$$
其中
$$ui(k, j′) = Di+k+1,j′−k−1 + w(i, i + j′)$$
只要證明 ui(k, j′) 有凹單調性,便可用前面所講的方法做到 $O(N^2 lg N)$ 的複雜度。
\subsubsection{2D/1D 凸性優化}
相較於 2D/1D 凹性優化,2D/1D 凸性有更多的性質,因此有更好的優化方式。首先
2D/1D 凸性在驗證上也比較容易,我們有以下的引理:2D/1D 的 DP 式為
$$ Di,j = min_{0≤j<i} Di,k + Dk+1,j + w(i, j) ; Di,i = 0$$
如果 w(i, j) 為凸單調性的且 w(i, i + 2) ≥ max(w(i, i + 1), w(i + 1, i + 2)),則 Di,j 也會符合凸單調性。
而且 2D/1D 凸性還有一個強大的性質:令 K(i, j) 為使 Di,j 達到最小值的 k 值,即 k 為使
Di,k + Dk+1,j + w(i, j) 最小者,則有 K(i, j − 1) ≤ K(i, j) ≤ K(i + 1, j)。
也因為以上定理,我們 DP 時可以從 j − i = 1, 2, · · · 的順序開始 DP。當我們在求 DPi,j 時
我們只需枚舉 K(i, j − 1) 到 K(i + 1, j) 即可,故對於所有 j − i = c的狀態,將他們都求出
所需枚舉的狀態數只有:
$$\sum_{0≤i<N−c} K(i + 1, j) − K(i, j − 1) = K(N − c, N) − K(0, c − 1) + N − c = O(N)$$
故求出所有 DP 值只需要 $O(N^2)$。

\subsection{插頭 DP}
根據某篇文章,他又稱作「基於連通性狀態壓縮的動態規劃」。
插頭 DP 通常用來處理 m × n 方格的一些組合計數,尤其是計算不同的某種特定路徑覆蓋的
方法數。
就拿最基本的來舉例,假設我們有一個 m × n 的方格,有些方格可以通行有些不能,求用數
個漢米頓圈覆蓋所有可以通行的方法樹有多少種?
而插頭 DP 的想法即是從右到左,從下到上依序討論每一個方格內路徑的樣式,即路徑要從
方格的上下左右哪個相鄰方格進來/出去。雖然說我們要求漢米頓圈覆蓋覆蓋的方式,但我
們可以想像成,我們先用路徑的方式討論,只要最後能把路徑「接起來」即可。
如圖所示,假設我們正在討論內部為交叉線條的那個方格,而我們 DP 的狀態便紀錄那
條粗線的每個線段,是否有「插頭」,即是否有路徑通過那個位置,而當我們要遞推到下一
格時,可以知道格子內路徑的樣式可以為那些只和格子右邊以及下邊的插頭有關,因此我們
只需枚舉一些情況就可以了,而如果格子是不能通行的,那相當於右邊以及下邊都不能有插
頭。
因此數個漢米頓圈覆蓋我們只需用 2 進位即可儲存狀態,但隨著題目越複雜,存的狀態可能
也會越複雜,比如題目如果要求只用 1 個漢米頓圈覆蓋,那就需要用 3 進制儲存。詳細的話
有興趣可以研究看看。


\section{特殊解題技巧}

\subsection{Meet in the middle}
HOJ 132 給你最多 30 個物品,每個物品有價值 c($−10^9$ ≤ c ≤ $10^9$),問是否存在一個集合 S 滿足 $\sum_{i\in S}c_i = 0$。
很簡單的想法是 $O(2^n)$ 爆搜,然後亂剪枝。但這實在不是很營養的辦法,那所以 有沒有複雜度合理一點的算法呢?\\
不難發現任何一個集合都是從前 15 個物品中找一些、後 15 個找一些,然後合併出來。所以我們可以把前 15 個跟後 15 個的所有子集合 $(2^15 + 2^15)$ 算出來,然後對於前半的每種子集合,搜尋看說後半的每種子集合中是否有集合跟他加起來是0 的,這明顯把後半每個子集合的權重都丟到平衡樹裡,就可以 $O(lgn)$ 的時間完成搜尋。
這樣子的時間複雜度是 $O(2^{n\over2} × 2 + 2^{n\over2} × lg2^{n\over2} )$,這最大也就 500000 左右而已,相當理性的計算量。

\subsection*{Exercises!!}
1. Balanced Cow Subset (USACO open 2012 gold)
給你 n(≤ 20) 隻乳牛,每隻乳牛有他的產乳量 mi,你現在要把這些乳牛分成黑牛、棕牛和白牛,滿足黑牛的總產乳量 = 棕牛的產乳量,問有幾種"把牛分到白牛的方法"。

\subsection{啟發式合併}
HOJ 191 給你一顆有根樹 (100000 個節點),某些葉子上有重量為 wi 的寶藏,對於每個非葉節點輸出它的有寶藏的葉子小孩們重量差距最小是多少。\\
第一眼應該會想到一個解法,對於每個節點把她所有寶藏葉子小孩排序好,然後掃過去計算答案。但這複雜度是很不妙的 $O(N^2)$。\\

那這時應該可以想到了一個優化,對於一個節點,你要計算的其實只是他各個寶藏葉子小孩對其他顆子樹的最小差距,也就是找他在某顆子樹的 lower bound 或是 upper bound 跟他的差距取 min。
不意外的話你現在有一個做法是這樣:\\
對於一個點依序把每顆子樹的所有寶藏小孩加進自己的集合,但加進去之前不忘更新這個點的答案。\\
為了方便取 upper bound 跟 lower bound 所以會用平衡樹當作自己的" 集合"的資料結構,所以複雜度會變成 $O(N^2 lgN)$,更不妙了。\\
那到底怎麼辦 OAOOO。\\
走投無路的你開始胡思亂想了。如果我每次合併平衡樹的時候是由小顆的一個一個點加進大顆的樹,那複雜度應該會好一點吧....\\
到底會好到多少呢? 你發現你的複雜度其實就是$"N× 插入的次數 × 插入一次的時間"$,所以每個寶藏小孩插入別人的次數是多少呢? 因為每次都是由小的插到大的,所以假如這次某隻寶藏小孩所在平衡樹的大小是,
那插入後所在的平衡數大小至少是 2x,而且如果這隻寶藏小孩所在的平衡數大小就是 N話,
那他在也不會去插入任何人了 (因為他已經在最大的平衡樹裡了),所以他只會待過 lgN 顆平衡樹。所以總複雜度就變成了 $O(Nlg^2 N)$,真是神奇。
\subsection*{Exercises!!}
1. Tree rotations (POI XVIII)
給你一顆二元樹,滿足每個非葉節點都有兩個小孩,
每個葉子都有一個 1 ~ n的數字且互不相同,讓 S 為所有葉子以前序表示法排列而成的序列,
你現在可以對這棵二元樹任一個非葉節點對調他的左子樹跟右子樹,
問你最小要調換幾次才能使新的 S 的逆序數對數最少。

\subsection{莫隊算法}
神奇大陸人發明的演算法。\\
通常我們再寫RMQ問題的時候,都要寫複雜的資料結構,
對於時間不多的資訊競賽來說,可能是造成你失敗的大坑,
因為你可能身陷其中而無法自拔,導至最後甚麼都沒寫出來。\\
莫隊算法提供了一個不一樣的作法。基本上他只能用來解靜態的區間查詢問題。
比如問你這個區間的最大值,或是區間的眾數。
我們可以分成兩個部份,第1是把問題轉換成你有一個區間,你可以從左右兩邊擴大或縮小區間。
接著我們把所有查詢離線,在依照某個順序處理。\\
那個順序就是,把序列切成$\sqrt N$塊,把左界在同一塊的查詢分成一組,然後每組再依照右界排序。
這樣的話,我們依照順序作查詢時,我們的同一塊裡面,左界會在$\sqrt N$的區間內抖動,右界會遞增,
複雜度變成$O(\sqrt N * (Q+N))$,Q是詢問次數。
因為我們的左界有兩種狀況,分別是抖動和移到下一格,假設我們K個分成一塊好了,
那抖動複雜度是$O(Q*K)$,而移動到下一格的複雜度,因為有N/K塊,,一次最多移動2*K,所以就是$O(N/K*2*K) = O(N)$。而右界在每一塊裡面都是遞增,複雜度是$O(N/K*N)$,最後複雜度就是$O(Q*K+N+N/K*N)$,我們發現取$K=\sqrt N$好像是個不錯的辦法。
莫隊的重點是要可以離線、不會有修改、可以擴大或縮小區間,上面的複雜度是基於擴展和縮小複雜度是$O(1)$,如果你用了甚麼set還是map的話就還會再往上跳。
\subsection*{Exercises!!}
TIOJ 1699 Problem I 害蟲決戰時刻\\
給你一堆蟲的編號,多筆詢問,問你一個區間內有沒有任何一種蟲出現的次數多於「區間長/K」。


\subsection{拉斯維加斯和蒙地卡羅演算法}
這兩種演算法是很有名的賭博演算法,被廣泛應用在喇分跟現代物理學的運算上。\\
當我們在做一些演算法時,我們其實有機率一次就作對,但是也有機率會錯。
拉斯維加斯是一種賭博演算法,他賭的不是答案的正確性,而是使用的資原。
你的問題通常會表現成"YES"和"NO",但是其中一種是有機率錯的,比如YES的正確率只有1/2之類的。
這種情況下,你可以在NO發生時就停止演算法,進而減少使用的資源,比如快速排序法。\\

然而有時候,我們沒有辦法算完全部的狀況,這時後就可以用萌地卡羅演算法,這個演算法主要是說,
我們用隨機抽樣來統計結果,結果會愈來愈接近我們的答案。
搭配上第1種演算法,我們可以得出,因為NO絕對是錯的,而YES真的是YES的機率P,
那連續YES兩次就是P*P,N次就是$P^N$,
所以我們可以多做幾次,直到錯誤率達到我們想要的標準,或是資源用盡。
\subsection*{Exercises!!}
STEP5 0129 驗算\\
給你很多個矩陣乘法的題目和結果,請你驗算這些矩陣乘法有沒有算錯。


\subsection{摹擬退火演算法}
有點算是萌地卡羅的一種應用,通常用在物理學和計算幾何上。\\
金屬晶體在降溫得過程中會不斷隨機跳動,慢慢得退進能量最低的晶格中,最後產生能量最低的結果。
而摹擬退火就是摹擬這個過程。
如果我們有很多點在平面上,我們希望找出一個點,到所有點的距離和最小,我們一開始一定是想到爆搜拉、三分搜之類的,但是那些都太麻煩了。
我們可以利用摹擬退火,先選一個點,寫一個估價函數(這裡就是計算到所有點的距離和),設定一個長度(也就是現在的能量,之後會慢慢減少)。之後開始隨機亂跳,如果跳過去發現估價函數的值變大了,那就回來,不然就跳過去,同時能量會慢慢衰退(可以將距離乘以0.9之類的方式)。
這樣最後我們就可以找到一個不錯的解,至於結果取決於能量衰退的速度和精度,如果衰退太快,可能會來不及跳到最佳解,太慢又會太浪費資源,我們只要讓能量衰退到一定的標準(通常跟題目要求的精度有關係),就可以結束掉演算法。



\end{document}