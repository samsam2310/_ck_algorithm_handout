%----------------------------------------------------------------------------------------
%   PACKAGES AND OTHER DOCUMENT CONFIGURATIONS
%----------------------------------------------------------------------------------------

\documentclass{article}

\usepackage{fancyhdr} % Required for custom headers
\usepackage{lastpage} % Required to determine the last page for the footer
\usepackage{extramarks} % Required for headers and footers
\usepackage[usenames,dvipsnames]{color} % Required for custom colors
\usepackage{graphicx} % Required to insert images
\usepackage{listings} % Required for insertion of code
\usepackage{caption}
\usepackage{courier} % Required for the courier font
\usepackage{lipsum} % Used for inserting dummy 'Lorem ipsum' text into the template
\usepackage{titlesec}

% My addon
\renewcommand{\thepage}{\Roman{page}}% Roman numerals for page counter
\usepackage{fontspec}   %加這個就可以設定字體
\usepackage{xeCJK}       %讓中英文字體分開設置
\setmainfont{Arial}
% \setCJKmainfont{思源黑體} %設定中文為系統上的字型,而英文不去更動,使用原TeX字型
\setCJKmainfont{WenQuanYi Micro Hei} %設定中文為系統上的字型,而英文不去更動,使用原TeX字型
\XeTeXlinebreaklocale "zh"             %這兩行一定要加,中文才能自動換行
\XeTeXlinebreakskip = 0pt plus 1pt     %這兩行一定要加,中文才能自動換行

% \def\footnotesize{\fontsize{16}{24}\selectfont}
\def\small{\fontsize{10}{15}\selectfont}
\def\normalsize{\fontsize{12}{16}\selectfont}
\def\large{\fontsize{16}{24}\selectfont}
\def\Large{\fontsize{20}{30}\selectfont}
\def\LARGE{\fontsize{24}{36}\selectfont}
\def\huge{\fontsize{32}{48}\selectfont}
\def\Huge{\fontsize{36}{54}\selectfont}

% Margins
\topmargin=-0.45in
\evensidemargin=0in
\oddsidemargin=0in
\textwidth=6.5in
\textheight=9.0in
\headsep=0.25in

\linespread{1.1} % Line spacing

% Set up the header and footer
\pagestyle{fancy}
\lhead{\hmwkTitle} % Top left header
% \chead{\hmwkClass\ (\hmwkClassInstructor\ \hmwkClassTime): \hmwkTitle} % Top center head
\rhead{\hmwkClass} % Top right header
% \lfoot{\lastxmark} % Bottom left footer
\cfoot{\thepage} % Bottom center footer
% \rfoot{Page\ \ of\ \protect\pageref{LastPage}} % Bottom right footer
\renewcommand\headrulewidth{0.4pt} % Size of the header rule
% \renewcommand\footrulewidth{0.4pt} % Size of the footer rule

\setlength\parindent{0pt} % Removes all indentation from paragraphs

%----------------------------------------------------------------------------------------
%   CODE INCLUSION CONFIGURATION
%----------------------------------------------------------------------------------------

\newcounter{nalg} % defines algorithm counter for chapter-level
\DeclareCaptionLabelFormat{algocaption}{\normalsize\bf Algorithm \thenalg} % defines a new caption label as Algorithm x.y

\lstnewenvironment{algorithm}[1][] %defines the algorithm listing environment
{   
    \refstepcounter{nalg} %increments algorithm number
    \captionsetup{labelformat=algocaption,labelsep=colon} %defines the caption setup for: it ises label format as the declared caption label above and makes label and caption text to be separated by a ':'
    \lstset{ %this is the stype
        frame=tB,
        numbers=left, 
        numberstyle=\normalsize,
        basicstyle=\normalsize, 
        keywordstyle=\color{black}\bfseries\em,
        keywords={,input, output, return, datatype, function, in, if, else, foreach, while, begin, end,} %add the keywords you want, or load a language as Rubens explains in his comment above.
        numbers=left,
        xleftmargin=.04\textwidth,
        #1 % this is to add specific settings to an usage of this environment (for instnce, the caption and referable label)
    }
}
{}


%----------------------------------------------------------------------------------------
%       NAME AND CLASS SECTION
%----------------------------------------------------------------------------------------

% 這裡記得改
\newcommand{\hmwkTitle}{培訓-6} % Assignment title
\newcommand{\hmwkDueDate}{2015年10月26日(月曜日)} % Due date
\newcommand{\hmwkClass}{圖論 II} % Course/class
\newcommand{\hmwkAuthorName}{joe59491} % Your name

% 不需要修改
\newcommand{\hmwkClassTime}{不需要修改} % Class/lecture time
\newcommand{\hmwkClassInstructor}{不需要修改} % Teacher/lecturer

%----------------------------------------------------------------------------------------
%       TITLE PAGE
%----------------------------------------------------------------------------------------

\title{\hmwkClass}
\author{\hmwkAuthorName}
\date{\hmwkDueDate}

%----------------------------------------------------------------------------------------

\begin{document}
\LARGE~\\[4ex]
\centerline{\bf\hmwkClass}\large\\[2ex]\centerline{\hmwkAuthorName}\\[2ex]\centerline{\hmwkDueDate}\\
\normalsize


%----------------------------------------------------------------------------------------
% 從這裡開始寫

\section{Minimum Spanning Tree}
\subsection{生成樹}
Spanning tree,一顆包含圖上所有節點的樹,權重為所包含的邊的權重和。
\subsection{最小生成樹}
Minimum Spanning Tree,權重最小的生成樹
\begin{description}
\item[ 性質1.]設T是G=(V,E)的MST。若刪除e(u,v)$\in $T則T會被分解成兩顆子樹Ta,Tb。且Ta跟Tb分別是G的子圖的MST且不相交。\\
由上可知,MST含有最佳子結構屬性,因此可能可以用dynamic programming或greedy algorithm求得。
\item[ 性質2.]設T是G=(V,E)的MST。在所有非自環邊中權重最小的邊必定$\in $T。\\
由上可知,MST含有切割屬性,把所有非自環邊中權重最小的邊加入T是合法的。也就是這條定理給的我們MST的貪婪選擇屬性。
\end{description}
\subsection{Prim's Algorithm}
Prim的演算法存任意一個節點開始,一步一步的擴展生成樹,每次選擇從樹上點連到非樹上點中權重最小的邊做擴張,直到所有節點都被加到樹上為止。\\
實作上有兩種複雜度$O(ElogV),O(EV^2)$。
\newpage
\begin{algorithm}[caption={Prim's Algorithm}, label={alg1}]
Graph Minimum_Spanning_Tree::Prim(G)
    T=G.V[0];
    while(G.V==T.V)
        e=min(w(T.V,G.V-T.V));
        T+=e;
    return T;
\end{algorithm}
\subsection{Kruskal's Algorithm }
Kruskal的演算法是將所有邊按權重排序,再由小到大加入,並檢查會不會在樹上形成環,如果會則跳過,否則就加入樹裡。\\
此方法的正確性仰賴切割性質。\\
實作上可以用disjoint set來檢查是否形成環,時間複雜度O(ElogE+E×α(V)) 。\\
\begin{algorithm}[caption={Kruskal's Algorithm}, label={alg1}]
Graph Minimum_Spanning_Tree::Kruskal(G)
    Graph T=NULL;
    Queue Q=sort(G.E);
    while(G.V==T.V)
        if(Tree_ID(Q[0].u)!=Tree_ID(Q[0].u))
            T+=Q[0];
        Q.pop_front();
    return T;
\end{algorithm}

\section{Single Source Shortest Path}
如何高效計算單個起點到圖上其他點的最短路徑,也就是單源最短路徑。\\
\subsection{BFS/DFS}
特殊圖(如樹、DAG),或是每條邊的權重都依樣則可使用。\\
\subsection{Dijkstra's Algorithm}
Dijkstra和Prim很像,Prim每次尋找不在樹上距離樹最近的點,而Dijkstra則是尋找不在樹上距離起點最近的點擴張。\\
Dijkstra的正確性仰賴三角不等式,若圖上存在邊權是負的會導致三角不等式不成立,因此Dijkstra演算法只能處理邊權皆為正的圖,時間複雜度O(ElogV)。\\
\begin{algorithm}[caption={Dijkstra's Algorithm}, label={alg1}]
Array Single_Source_Shortest_Path::Dijkstra(G,s)
    Array d(G.V)=INFINITE;
    d[s]=0;
    Heap H(s);
    while(!H.empty())
        v=H.pop_top();
        for(e:G[v].E)
            if(d[e.u]>d[v]+e.w)
                d[e.u]=d[v]+e.w;
                Heap.push_replace(e.u,d[e.u]);
    return d;
\end{algorithm} 
\subsection{Bellman-Ford Algorithm}
Dijkstra雖然快,卻不能處理有負權的圖,對於存在負權的圖我們常常採用Bellman-Ford演算法。\\
我們先定義一個動作較鬆弛(relax):若d[v]代表起點到v的最短路徑,若存在點u使得d[u]>d[v]+w[u][v]則我們可以更新d[u],這個動作就稱為用v鬆弛u。\\
我們可以稍微修改Dijkstra算法的策略,把針對節點鬆弛改為針對邊鬆弛,若每條邊都考慮一次後,沒有任何節點被鬆弛則說明瞭所有最短路徑都被找出,否則就要在考察一次所有邊。\\
而我們可以證明對於所有不存在負環的圖,都可以在$|V|-1$輪內找出所有最短路徑,無論是否存在負權邊。若重複了$|V|-1$次後仍然有邊可鬆弛則代表圖上有負環,因此Bellman-Ford演算法也可用在判斷圖上有無負環,時間複雜度O(VE)。\\
\newpage
\begin{algorithm}[caption={Bellman-Ford Algorithm}, label={alg1}]
Array Single_Source_Shortest_Path::Bellman_Ford(G,s)
    Array d(G.V)=INFINITE;
    d[s]=0;
    for(i in 1..G.V.size())
        Bool relax=0;
        for(e:G.E)
            if(d[e.u]>d[e.v]+e.w)
                d[e.u]=d[e.v]+e.w;
                relax=1;
        if(!relax)
            break;
        if(i==G.V.size())
            throw Error;
    return d;
\end{algorithm}    
\subsection{Shortest Path Fast Algorithm}
俗稱的SPFA演算法只是Bellman-Ford的一個優化,基本想法是一個點必須被鬆弛過纔有必要用來鬆弛別的點。\\
此演算法曾由西南交通大學段凡丁《關於最短路徑的 SPFA 快速算法》重新發現,於是中文網路出現了 Shortest Path Faster Algorithm, SPFA 的通俗稱呼。學術上查無此稱呼,請勿濫用。\\
實作上可以用一個容器(例如Queue、Stack)來存放目前被鬆弛過的點,每次從容器中取一個點v來鬆弛與之相鄰的點u,若u成功被鬆弛了,則把u丟入容器中。\\
如同Bellman-Ford一樣,我們可以開一個陣列紀錄一條邊鬆弛了幾次,來判斷圖上有無負環。\\
時間複雜度O(VE),平均情況下期望時間複雜度為O(E)。\\
證明的網址:http://goo.gl/0UPjEF\\
因為複雜度不穩定,因此如果是在正權圖上,還是建議用複雜度較穩定的Dijstra。\\
\newpage
\begin{algorithm}[caption={Shortest Path Fast Algorithm}, label={alg1}]
Array Single_Source_Shortest_Path::SPFA(G,s)
    Array relax(G.E)=0;
    Array d(G.V)=INFINITE;
    d[s]=0;
    Queue Q(s);
    while(!Q.empty())
        v=Q.pop_top();
        for(e:G[v].E)
            if(d[e.u]>d[v]+e.w)
                d[e.u]=d[v]+e.w;
                if(++relax[e]==G.V)
                    throw Error;
                if(!Q.find(u))
                    Q.push(u);
    return d;
\end{algorithm}

\section{All Pairs Shortest Path}
找出圖G上所有點對的最點距離,稱為全點對最短路徑問題。\\
我們當然可以在圖上運行V次Single Source Shortest Path,但是要嘛不能處理負邊,要嘛複雜度太高,要嘛複雜度不穩定。\\
因此我們想要有一個更加簡潔且複雜度漂亮的演算法。\\
\subsection{Floyd-Warshall Algorithm}
Floyd-Warshall演算法使用DP的技巧,我們定義dp[k][i][j]代表節點i到j只通過集合{0..k}的最短路徑大小。\\
因此:dp[k][i][j]=min(dp[k-1][i][j],dp[k-1][i][k]+dp[k-1][k][j]);\\
實作上dp表格可以用滾動的方式節省記憶體,時空複雜度$O(V^3)$。\\
\newpage
\begin{algorithm}[caption={Floyd-Warshall Algorithm}, label={alg1}]
Array<Array> All_Pairs_Shortest_Path::Floyd_Warshall(G)
    Array<Array> dp(G.V,G.V)=INFINITE;
    for(k:G.V)
        for(i:G.V)
            for(j:G.V)
                dp[i][j]=min(dp[i][j],dp[i][k]+dp[k][j]);
    return dp;
\end{algorithm}
\subsection{Johnson's Algorithm}
如果圖上沒有負邊,那麼運行|V|次Dijkstra的時間成本可能優於Floyd-Warshall。那麼是否存在有負權的情況下也能使用這個優秀演算法的方法呢?\\
Johnson的演算法可以將圖上的負權轉換為正權,我們只需先運行一次Johnson演算法把圖G轉為正權圖,接者就可以使用Dijkstra求出全點對最短路徑了。\\
Johnson演算的的精隨是:\\
對於每個節點賦予他一個權值h[v],且對每條邊賦予新權重w'[u][v]=w[u][v]+h[u]-h[v]>=0後得到新圖G'。\\
這樣在G'上求出來的距離是:\\
d'[s][t]=w'[s][v[1]]+w'[v[1]][v[2]]+...+w'[v[k],t];\\
d'[s][t]=w[s][v[1]]+h[s]-h[v[1]]+w[v[1]][v[2]]+h[v[1]]=h[v[2]]...+w[v[k],t]+h[v[k]]-h[t];\\
d'[s][t]=w[s][v[1]]+w[v[1]][v[2]]+...+w[v[k],t]+h[s]-h[t];\\
d'[s][t]=d[s][t]+h[s]-h[t],d'[s][t]>=0;\\
也就是說我們只要求出d'也就能在常數時間內轉換出d,因為G'是正權圖,所以可以使用Dijkstra。\\
剩下的問題就是h的求法,而這正是Johnson演算法的精妙之處,他發現只要將所求的w[u][v]+h[u]-h[v]>=0移項後就會變成h[v]<=h[u]+w[u][v],也就是說h[v]的其中一個解就是d[v],而d[v]只需用一次Bellman-Ford就可求得。\\
那我們要用哪一個起點求得d[v]呢?若是用了不好的起點,可能會求不出所有的d[v],因此Johnson想出了一個方法就是重新設立一個超級原點s,並對圖上所有點u新增連到的邊,且邊權為0,這樣就能保證可以對所有節點求出d[v]了。\\
總時間複雜度是O(VE+VElogV)在稀疏圖上比Floyd-Warshall更佳。\\

\section{二分圖最大匹配}
\begin{description}
\item[ 1.匹配:]在圖論中,一個「匹配」(matching)是一個邊的集合,其中任意兩條邊都沒有公共頂點。
\item[ 2.最大匹配:]一個圖所有匹配中,所含匹配邊數最多的匹配,稱爲這個圖的最大匹配。
\item[ 3.完美匹配:]如果一個圖的某個匹配中,所有的頂點都是匹配點,那麼它就是一個完美匹配。
\end{description}
\subsection{匈牙利演算法}
\begin{description}
\item[ 1.交替路:]從一個未匹配點出發,依次經過非匹配邊、匹配邊、非匹配邊…形成的路徑叫交替路。
\item[ 2.增廣路:]從一個未匹配點出發,走交替路,如果途徑另一個未匹配點(出發的點不算),則這條交替路稱爲增廣路(agumenting path)。
\end{description}
增廣路有一個重要特點,非匹配邊比匹配邊多一條。因此,研究增廣路的意義是改進匹配。只要把增廣路中的匹配邊和非匹配邊的身份交換即可。由於中間的匹配節點不存在其他相連的匹配邊,所以這樣做不會破壞匹配的性質。交換後,圖中的匹配邊數目比原來多了1條。\\
我們可以通過不停地找增廣路來增加匹配中的匹配邊和匹配點。找不到增廣路時,達到最大匹配(這是增廣路定理)。\\
\newpage
\begin{algorithm}[caption={Hungary}, label={alg1}]
Int Matching::Hungary(G)
    Array<Node> go=NULL;
    Array<Bool> used;
    Int ans=0;
    for(Node i:G.X)
        used=0;
        Function F=(Node now)->Bool{
            for(Node j:G[now])
                if(!used[j]++&&(go[j]==NULL||F(go[j])))
                    go[j]=now;
                    return 1;
            return 0;
            };
        ans+=F(i);
    return ans;
\end{algorithm}  
\subsection{Hopcroft-Karp Algorithm}
匈牙利演算法的改進版本。\\
在匈牙利算法中,我們每次尋找一條增廣路來增加匹配集合M。可以證明,每次找增廣路的複雜度是O(E),一共需要增廣O(V)次,因此總時間複雜度為O(VE)。\\
為了降低時間複雜度,在Hopcroft-Karp算法中,我們在增加匹配集合M時,每次尋找多條增廣路。可以證明,這樣迭代次數最多為2sqrt(V),所以時間複雜度就降到了O(sqrt(V)E)。\\
\subsection{最大權二分匹配問題}
最大權二分匹配問題就是給二分圖的每條邊一個權值,選擇若干不相交的邊,得到的總權值最大。\\
\subsection{Kuhn-Munkres Algorithm}
解決這個問題可以用Kuhn-Munkres算法(簡稱KM)。理解KM算法需要首先理解“可行頂標”的概念。可行頂標是指關於二分圖兩邊的每個點的一個值lx[i]或ly[j],保證對於每條邊w[i][j]都有lx[i]+ly[j]-w[i][j]>=0。如果所有滿足lx[i]+ly[j]==w[i][j]的邊組成的導出子圖中存在一個完美匹配,那麼這個完美匹配肯定就是原圖中的最大權匹配。理由很簡單:這個匹配的權值之和恰等於所有頂標的和,由於上面的那個不等式,另外的任何匹配方案的權值和都不會大於所有頂標的和。\\
但問題是,對於當前的頂標的導出子圖並不一定存在完美匹配。這時,可以用某種方法對頂標進行調整。調整的方法是:根據最後一次不成功的尋找交錯路的DFS,取所有i被訪問到而j沒被訪問到的邊(i,j)的lx[i]+ly[j]-w[i][j]的最小值d。將交錯樹中的所有左端點的頂標減小d,右端點的頂標增加d。經過這樣的調整以後:原本在導出子圖裏面的邊,兩邊的頂標都變了,不等式的等號仍然成立,仍然在導出子圖裏面;原本不在導出子圖裏面的邊,它的左端點的頂標減小了,右端點的頂標沒有變,而且由於d的定義,不等式仍然成立,所以他就可能進入了導出子圖裏。\\
初始時隨便指定一個可行頂標,比如說lx[i]=max{w[i][j]|j是右邊的點},ly[i]=0。然後對每個頂點進行類似Hungary算法的find過程,如果某次find沒有成功,則按照這次find訪問到的點對可行頂標進行上述調整。這樣就可以逐步找到完美匹配了。\\
值得注意的一點是,按照上述d的定義去求d的話需要$O(N^2)$的時間,因爲d需要被求$O(N^2)$次,這就成了算法的瓶頸。可以這樣優化:設slack[j]表示右邊的點j的所有不在導出子圖的邊對應的lx[i]+ly[j]-w[i][j]的最小值,在find過程中,若某條邊不在導出子圖中就用它對相應的slack值進行更新。然後求d只要用O(N)的時間找到slack中的最小值就可以了。\\
\begin{algorithm}[caption={Kuhn–Munkres Algorithm}, label={alg1}]
Int Matching::Kuhn_Munkres(G)
    Array<Node> go=NULL;
    Array<Int> lx=0,ly=0,sl;
    Array<Bool> usx,usy;
    Function F=(Node now)->Bool{
        usx[now]=1;
        for(y:G[now])
            if(usy[y])continue;
            Int v=lx[now]+ly[y]-G.w(now,y);
            if(v)sl[y]=min(sl[y],v);
            else
                usy[y]=1;
                if(go[y]==NULL||F(go[y]))
                    go[y]=now;
                    return 1;
        return 0;
        };
    for(x:G.X)
        sl=Infinite;
        while(1)
            usx=0;
            usy=0;
            usx[x]=1;
            if(F(x))break;
            Int d=Infinite;
            for(y:G.Y)
                if(!usy[y])
                    d=min(d,sl[y]);
            for(z:G.X)
                if(usx[z])
                    lx[z]-=d;
            for(y:G.Y)
                if(usy[y])
                    ly[y]+=d;
                else
                    sl[y]-=d;
    int ans=0;
    for(y:G.Y)
        ans+=G.w(go[y],y);
    return ans;
\end{algorithm}             

\section{Tree}
這裡講解一些進階的樹概念。\\
\subsection{重心}
\begin{description}
\item[ 1.定義:]將其拔掉之後剩餘的子樹中節點最多的子樹節點最少。
\item[ 2.性質:]可以保證拔掉重心之後,剩下的樹盡量平衡,因此可用來作樹分治。
\end{description}
\subsection{LCA}
最近公共祖先:Lowest Common Ancestor,給定一個有根樹T時,對於任意兩個節點u,v,找到一個離根最遠的節點x,使得x同時為u跟v的祖先,則x就是u跟v的最近公共祖先。\\
\subsection{倍增法}
先預處理出每個點的深度跟子樹大小,還有二次方父親。查詢(a,b)時先讓兩個點跳至相同深度後,先檢查兩個點是否是同一個點如果是的話,則此點是LCA。\\
若不是的話,檢查兩個點同時向上$2^x$個父親,如果是同一人的話,則不要跳,否則就往上跳,接者把x--,重複做直到x<0,然後接者兩個點的父親就是LCA了。\\
\begin{algorithm}[caption={倍增法}, label={alg1}]
Void Tree::LCT_Pretreatment(Node now,Node father)
    now.father[0]=father;
    now.depth=father.depth+1;
    for(Int i=1;i<=lg(now.depth);i++)
        now.father[i]=now.father[i-1].father[i-1];
    for(Node x:now.kids)
        LCT_Pretreatment(x,now);
Node Tree::LCA(Node a,Node b)
    if(a.depth>b.depth)
        swap(a,b);
    for(Int x=b.depth-a.depth,y=x&-x;y;x^=y,y=x&-x)
        b=b.father[lg(y)];
    if(a==b)
        return a;
    for(Int i=lg(a.depth);i>=0;i--)
        if(a.father[i]!=b.father[i])
            a=a.father[i];
            b=b.father[i];
    return a.father[0];
\end{algorithm}
\subsection{轉換成RMQ}
把樹壓平之後,你會發現變成在一個區間內求最小編號的位置,而那個位置代表的節點就是LCA。\\
\subsection{樹鍊剖分}
\begin{description}
\item[ 1.樹練:]樹上的路徑。
\item[ 2.剖分:]把路徑分為重鍊、輕鍊。
\item[ 3.兒子:]一個點的子結點。
\item[ 4.重兒子:]兒子中子樹大小最大的點。
\item[ 5.輕兒子:]重兒子以外的其他兒子。
\item[ 6.重邊:]連到重兒子的邊。
\item[ 7.輕邊:]連到輕兒子的邊。
\item[ 8.重鍊:]由重邊形成的路徑。
\item[ 9.輕鍊:]輕邊。
\end{description}
\begin{description}
\item[ 性質1.]如果(u,v)是輕邊,則v的子樹大小*2<u的子樹大小。\\
\item[ 性質2.]從根到某一點路徑上的輕鍊、重鍊個數都不大於logN,我們可以把每個重鍊用一顆線段樹維護,這樣我們就能在樹上做鍊的修改,查詢。\\
\end{description}
\subsection{樹壓平EX}
我們之前樹壓平的時候,進入兒子的順序是隨便的,現在我們改成優先進入重兒子,則我們會發現樹壓平之後,每條重鍊都會是在一個區間之內,這樣我們就能同時子樹、樹鍊修改跟查詢。
\subsection{樹心剖分}
當我們坐樹分治的時候,我們可以把由重心所構成的樹建出來,而這個樹的深度不會超過logN,而可以利用這個性質做到一些有趣的事情,例如支援單點修改,跟查詢一個點到整棵樹的距離乘權重和等等。

\section{集}
\subsection{支配集(dominating set)}
D是G的支配集:$\forall v\in(G.V\backslash D),\exists u\in D,(u,v)\in G.E$\\
最小支配集(minimum dominating set):頂點數最少\\
從完美匹配中的每條邊取一個端點構成一個支配集?(O)\\
從最大匹配中的每條邊取一個端點構成一個支配集?(X)\\
定理:無孤立點圖中,存在支配集\\
此為NP-hard問題\\
貪心算法:每一輪遞迴總是選取能支配最多剩餘頂點的那個頂點,近似比:1+logν。\\
\subsection{點獨立集(vertex independent set)}
I是G的點獨立集:$\forall u,v\in I,(u, v)\notin G.E$\\
最大點獨立集(maximum vertex independent set):頂點數最大\\
定理:最大點獨立集=圖中點的個數-最大匹配數。\\
最大點獨立集=補圖中的最大團,此為NP-hard問題\\
但當有最大二分匹配時,求最大點獨立集的時間複雜度等同於一次 Graph Traversal 的時間。\\
\subsection{點覆蓋集(vertex cover)}
F是G的點覆蓋集:$\forall (u,v)\in G.E,{u,v}\cap F\neq \emptyset $\\
最小點覆蓋集(minimum vertex cover):頂點數最小\\
連通圖中,點覆蓋集必為支配集\\
定理:F是點覆蓋集當且僅當$G.V\backslash F$是點獨立集\\
一樣是NP-hard問題\\
但是在二分圖上可以利用二分圖匹配完成(為點獨立集的補集)\\
\subsection{邊獨立集(edge independent set)}
就是匹配(兩兩不相鄰的邊)\\
最大邊獨立集(maximum edge independent set):最大匹配\\
一般圖:不存在近似比好於1.3606的多項是時間算法(除非P=NP)\\
二分圖:參見二分圖匹配\\
\subsection{邊覆蓋集(edge cover)}
L是G的邊覆蓋集:$\forall u\in G.V,\exists v\in G.V,(u,v)\in L$\\
最小邊覆蓋集(minimum edge cover):邊數最少\\
定理:最小邊覆蓋集=圖中點的個數-最大匹配數。\\
最大邊獨立集、最小邊覆蓋,兩者幾乎相等,差異是未匹配點所連接的邊。\\
\subsection{總結}
最小邊覆蓋數=最大獨立集數=G.V-最大匹配數\\
最大團=最大獨立集的補圖\\
最小點覆蓋=最大匹配數\\
最小點覆蓋+最大獨立數=G.V\\
最小割=最小點權覆蓋集=點權和-最大點權獨立集\\

\end{document}