%----------------------------------------------------------------------------------------
%	PACKAGES AND OTHER DOCUMENT CONFIGURATIONS
%----------------------------------------------------------------------------------------

\documentclass{article}

\usepackage{fancyhdr} % Required for custom headers
\usepackage{lastpage} % Required to determine the last page for the footer
\usepackage{extramarks} % Required for headers and footers
\usepackage[usenames,dvipsnames]{color} % Required for custom colors
\usepackage{graphicx} % Required to insert images
\usepackage{listings} % Required for insertion of code
\usepackage{caption}
\usepackage{courier} % Required for the courier font
\usepackage{lipsum} % Used for inserting dummy 'Lorem ipsum' text into the template
\usepackage{titlesec}

% My addon
\renewcommand{\thepage}{\Roman{page}}% Roman numerals for page counter
\usepackage{fontspec}   %加這個就可以設定字體
\usepackage{xeCJK}       %讓中英文字體分開設置
\setmainfont{Arial}
\setCJKmainfont{思源黑體} %設定中文為系統上的字型,而英文不去更動,使用原TeX字型
\XeTeXlinebreaklocale "zh"             %這兩行一定要加,中文才能自動換行
\XeTeXlinebreakskip = 0pt plus 1pt     %這兩行一定要加,中文才能自動換行

% \def\footnotesize{\fontsize{16}{24}\selectfont}
\def\small{\fontsize{10}{15}\selectfont}
\def\normalsize{\fontsize{12}{16}\selectfont}
\def\large{\fontsize{16}{24}\selectfont}
\def\Large{\fontsize{20}{30}\selectfont}
\def\LARGE{\fontsize{24}{36}\selectfont}
\def\huge{\fontsize{32}{48}\selectfont}
\def\Huge{\fontsize{36}{54}\selectfont}

% Margins
\topmargin=-0.45in
\evensidemargin=0in
\oddsidemargin=0in
\textwidth=6.5in
\textheight=9.0in
\headsep=0.25in

\linespread{1.1} % Line spacing

% Set up the header and footer
\pagestyle{fancy}
\lhead{\hmwkTitle} % Top left header
% \chead{\hmwkClass\ (\hmwkClassInstructor\ \hmwkClassTime): \hmwkTitle} % Top center head
\rhead{\hmwkClass} % Top right header
% \lfoot{\lastxmark} % Bottom left footer
\cfoot{\thepage} % Bottom center footer
% \rfoot{Page\ \ of\ \protect\pageref{LastPage}} % Bottom right footer
\renewcommand\headrulewidth{0.4pt} % Size of the header rule
% \renewcommand\footrulewidth{0.4pt} % Size of the footer rule

\setlength\parindent{0pt} % Removes all indentation from paragraphs

%----------------------------------------------------------------------------------------
%	CODE INCLUSION CONFIGURATION
%----------------------------------------------------------------------------------------

\newcounter{nalg} % defines algorithm counter for chapter-level
\DeclareCaptionLabelFormat{algocaption}{\normalsize\bf Algorithm \thenalg} % defines a new caption label as Algorithm x.y

\lstnewenvironment{algorithm}[1][] %defines the algorithm listing environment
{   
    \refstepcounter{nalg} %increments algorithm number
    \captionsetup{labelformat=algocaption,labelsep=colon} %defines the caption setup for: it ises label format as the declared caption label above and makes label and caption text to be separated by a ':'
    \lstset{ %this is the stype
        frame=tB,
        numbers=left, 
        numberstyle=\normalsize,
        basicstyle=\normalsize, 
        keywordstyle=\color{black}\bfseries\em,
        keywords={,input, output, return, datatype, function, in, if, else, foreach, while, begin, end,} %add the keywords you want, or load a language as Rubens explains in his comment above.
        numbers=left,
        xleftmargin=.04\textwidth,
        #1 % this is to add specific settings to an usage of this environment (for instnce, the caption and referable label)
    }
}
{}

%----------------------------------------------------------------------------------------
%	NAME AND CLASS SECTION
%----------------------------------------------------------------------------------------

\newcommand{\hmwkTitle}{培訓-3} % Assignment title
\newcommand{\hmwkDueDate}{2015年9月30日(火曜日)} % Due date
\newcommand{\hmwkClass}{二分搜尋と演算法設計の方法} % Course/class
\newcommand{\hmwkClassTime}{TTD} % Class/lecture time
\newcommand{\hmwkClassInstructor}{TT} % Teacher/lecturer
\newcommand{\hmwkAuthorName}{joe59491} % Your name


%----------------------------------------------------------------------------------------
%	TITLE PAGE
%----------------------------------------------------------------------------------------

\title{\hmwkClass}
\author{\hmwkAuthorName}
\date{\hmwkDueDate}

%----------------------------------------------------------------------------------------

\begin{document}
\LARGE~\\[4ex]
\centerline{\bf\hmwkClass}\large\\[2ex]\centerline{\hmwkAuthorName}\\[2ex]\centerline{\hmwkDueDate}\\
\normalsize


\section{Binary Search 二分搜尋}
\subsection*{Introduction}
現在給你一個已經照大小順序排好的數列,怎樣快速的找到x這個數字的位置(或是告知x不存在)?
\begin{description}
\item[ Step 1.]看一下數列正中間的值,如果剛好等於x代表你找到了
\item[ Step 2.]不是的話,想一下x會在左半邊還是右半邊,然後把沒有x的那半邊丟掉
\item[ Step 3.]如果數列還有東西,回到步驟1,不然代表找不到
\end{description}

~\\時間複雜度為 O(C lgN),C是檢查的時間複雜度,N是數列的大小。\\
以下虛擬碼代表在一個排序過的陣列中尋找x的位置的方法。

\begin{algorithm}[caption={Binary Search}, label={alg1}]
function Binary_Search(x, array[])
    l = 0
    r = n - 1 (n是array的大小)
    while l < r
        mid = (l + r) / 2
        if x > array[mid]
            l = mid + 1
        else
            r = mid
    return l
end function
\end{algorithm}


\subsection{常用的地方}
1.找東西\\
2.求符合某條件的最小(最大)值\\
3.奇怪的互動題\\


\subsection{相關STL}

\begin{description}
\item[ 1.]upper\_bound\\
最後一個小於val的值的位置。\\
template <class ForwardIterator, class T, class Compare>\\
ForwardIterator upper\_bound (ForwardIterator first, ForwardIterator last,\\
\hspace*{2em}const T\& val[, Compare comp]);
\item[ 2.]lower\_bound\\
最後一個小於等於val的值的位置。\\
template <class ForwardIterator, class T, class Compare>\\
ForwardIterator lower\_bound (ForwardIterator first, ForwardIterator last,\\
\hspace*{2em}const T\& val[, Compare comp]);
\item[ 3.]binary\_search\\
不常用(僅回傳bool),確認val是否存在。
\item[ 4.]equal\_range\\
相當於make\_pair(lower\_bound(...),upper\_bound(...));
\end{description}


\subsection{Exercises!!}

\begin{description}
\item[ 1.]<TIOJ 1208 第K大連續和>\\
一個總長度為項的數列S,其連續和共有n(n+1)/2個。請問,這n(n+1)/2個連續和之中,第k大的是多少?
\item[ 2.]<TIOJ 1839(IOI 2013) 洞穴Cave>\\
你迷路了。你發現一個入口。入口被連續N個門阻隔。有N個開關分別連接到這N個不同的門。\\
門依序由0~N-1編號,最接近你的那扇門編號為0。開關編號也是0~N-1但你並不知道哪個開關連到哪扇門。\\
開關都位於洞穴的入口處。每個開關可以被設定成上或下。每個開關只有一個方向是正確的。若一個開關被設定在正確的方向,則其連接到的門便會打開,反之則否。每一個開關的正確方向可能不相同,現在你並不知道開關的正確方向。\\
你可以將開關設定成任何的組合,並且走入這個洞穴觀察第一個未開的門。這些門都是不透明的,當你發現第一個未開的門時,你無法看到任何後方的門。\\
你可以嘗試最多70,000種組合。你的任務是把所有的門都打開。
\end{description} 



\section{Divide \& Conquer 分而治之}
\subsection*{Introduction}
大陸稱之為分治,有些奇怪的人稱之為Fun Jizz或Fun and Jizz,是一種常見的算法設計方法。\\
大致上的步驟為:
\begin{description}
\item[ Step 1.]切開~把問題切碎
\item[ Step 2.]處理~把碎片解決
\item[ Step 3.]合併~把碎片合併
\end{description}

\subsection{D\&C 的時間複雜度}
時間複雜度通常都是一個遞迴函式,推出一般項的方法有亂猜後用數學歸納法證明,劃出遞迴樹以及使用主定理(Master theorem)。有興趣可以自己去Google。

\subsection{常用的地方}
許多的算法都有運用到分治的想法,例如前面所說的二分搜、merge sort、快速矩陣乘法、最近點對等等...。\\
一般出現的時候,如果不是已經看過的類型,通常很難當場想出來。


\subsection{Exercises!!}

\begin{description}
\item[ 1.]<TIOJ 1080 逆序數對>\\
輸入一個數列,輸出數列中有多少數對是逆序的。
\item[ 2.]<TIOJ 1355 河內之塔-蘿莉塔>\\
有三根柱子,第一根上有圈圈,每個大小都不一樣,且大的在下面,然後請輸出如何把所有的圈圈移到第三根柱子上,過程中不能讓小的圈圈在大的圈圈下面。
\item[ 3.]<TIOJ 1500 平面最近點對>\\
給你平面上N個點,問最近的兩個點有多近。
\item[ 4.]<POJ 1741 樹分治>\\
給你一棵樹,問樹上距離小於等於K的點對數量。
\item[ 5.]<TIOJ 1631 點連接遊戲>\\
在正方形xy平面上給你N個紅點跟綠點(左上右上紅點,左下右下綠點),保證三點不共線,請用不交叉的數個線段分別把紅點跟綠點連起來。
\end{description}



\section{Dynamic Programming 動態規劃}
\subsection*{Introduction}
最佳化問題(Optimization Problem),是從諸多可行解中找出的最佳解(Optimal Solution)的問題。 \\
動態規劃(Dynamic Programming),簡稱DP,是解決每種最佳化問題的典型策略,許多的最佳化問題都要由一連串的決策導出,而當我們面對這些決策時,會不斷的出現擁有相似型態的子問題。\\

\subsection{What}
DP使用將原問題分解成子問題的方式推導出源問題的解答。我們常常把原問題分解成數個較容易解決的子問題,如果子問題還是過難的話,就繼續分解成更多的子問題。而當不同子問題的最佳解可能來自相同的子子問題的時候,次技巧精妙之處就是把處理過的解答存起來,以免重複計算,因此適合以DP解決的問題通常都有子問題重疊(Overlapping Subproblems)的性質,也就是不同子問題最佳解可能會包含相同的子子問題。

\subsection{When}但是也不是所有最佳化問題都能使用DP來解決,適合使用DP的最佳化問題也需要滿足幾個特性才能確保DP的正確性。
\begin{description}
\item[ 1.]最佳子結構\\
問題的最佳解中含有子問題 的最佳解,也就是說問題的最佳 解可以由子問題的最佳解推得。
\item[ 2.]無後效性\\
若子問題決定的母問題最佳 解會影響到子問題的最佳解,則 有後效性。
\end{description}

\subsection{How}
使用DP解決問題時,我們通常會先辨認出問題的結構特徵,然後遞迴更新各個最佳解,如果有需要則會將計算過的子問題存起來,方便重複取用而不必重複計算,從而提高效率。 
一般實作時主要分為兩種風格:
\begin{description}
\item[ 1.]Bottom-Up\\
從最小的子問題開始計算答案,逐漸算出更大子問題的答案,直到算出目標解為止。\\
優點:可以直接使用迴圈跑過整個表格計算出答案,且適當的設計迭代順序的話還可以壓縮記憶體使用量。\\
缺點:遇到狀態較多的題目表格會太大,可能會計算無用的子問題。
\item[ 2.]Top-Down\\
從原問題開始遞迴計算子問題,將算過的子問題最佳解存起來,並逐步地算出答案。\\
優點:不會計算到多於的子問題,且子問題數目很多時仍可使用。\\
缺點:表格大小會和子問題總數一樣大,呼叫函式遞迴也需要負擔額外的成本。
\end{description} 


\subsection{Exercises!!}

\begin{description}
\item[ 1.]<工廠生產問題>\\
一間工廠有兩條生產線,生產線上各有N個工作站,兩條生產線對應的工作站工作相同但是效率不同,你可以把一條生產線中的半成品轉到另外一條生產線上(需要成本),做每件事情的成本可能都不相同,問生產一件產品所需的最小成本是多少?
\item[ 2.]<最大正方形>\\
給你一個只含0跟1且大小為N*N的正方形,問最大只由1組成的方形可以多大?
\item[ 3.]<TIOJ 1029 A Game>\\
有一串由N個正整數所組成的數列,兩個玩者輪流拿走一個最左邊或最右邊的數,直到最後所有的數都取完之後,兩個玩者分別把自己所取到數加總,分數較高的人獲勝。問兩個玩家分別所能得到的最高分數是多少?
\item[ 4.]<切木棒>\\
你有一根長度N 的木棒,對於長度1,2...N的木棒,某商人願意畫a1,a2...N 的價格跟你購買,如果切木棒不花錢問你最多能賺多少錢?
\item[ 5.]<01背包問題>\\
你擁有一個載重V的背包,和N個物品,每個物品都有各自的重量和價值,問你在不超重的前提下,最多可以裝多少價值的物品?
\item[ 6.]<TIOJ 1816 石油>\\
給你一個n*m的矩陣,請從中選出三個互不覆蓋的且大小為k*k的正方形,並最大化總和。
\item[ 7.]<TIOJ 1019 兔子跳鈴鐺>\\
給你N個鈴鐺的水平位置,你每次可以從x跳到x+1或x+2的鈴鐺上,問從第1個跳到第n個鈴鐺所需的最小水平移動距離。
\item[ 8.]<TIOJ 1014 打地鼠>\\
有編號1,2,...,n的地鼠動。玩家站在最左邊,與第一個地鼠洞相距1公尺,準備開始這個遊戲。編號為 i 的地鼠洞每Ti秒地鼠會出現一次。被打的地鼠不再出現,所有地鼠打完就結束遊戲,問結束遊戲所需的最少秒數。
\item[ 9.]<codeforces 214E Relay Race>\\
現在你有一個n*n的矩陣,你要選兩條(1,1)->(n,n)的捷徑,讓兩條路徑聯集上的數字總和最大。
\item[ 10.]<Step5 0004 一切的開始>\\
有N隻殭屍A1,A2,...AN排成一排,殭屍分兩種,你想要讓所有殭屍都變第一種,你可以改變一隻的屬性,或是改變A1~Ax的屬性,問你最少要改幾次。
\item[ 11.]<Step5 0093 數Misaka妹妹>\\
你想要數數看排了多少御坂妹妹,但你卻發現自己每連續數到n個御坂妹妹眼花,如果眼前總共有m個人在排隊,那麼有幾種排列方式是不會算錯的呢?
\end{description}



\section{Greedy 貪婪}
\subsection{What}
Greedy使用一個很當純的想法尋找答案,就是每次只選擇當前看似最佳的選擇。換言之我們每次都選擇當前最佳解,並期望能達到全局最佳解。因此Greedy不一定能得到問題的最佳解,但是當他被證明可行時(否則他只能獲得近似解),Greedy通常也會是這個問題的最佳策略,因為他放棄搜尋所有解答空間,效率也因此提高很多。


\subsection{When}
原則上要符合兩個特性才能使用Greedy。
\begin{description}
\item[ 1.]最佳子結構\\
問題的最佳解中含有子問題 的最佳解,也就是說問題的最佳 解可以由子問題的最佳解推得。
\item[ 2.]貪婪選擇屬性\\
每次決策時我們可以選擇當前看似最佳的選擇,並在繼續Greedy下去。也就是說每次決策的時候只考慮目前為止做過的選擇,不考慮未來可能要做的選擇或是子問題的解答。
\end{description}


\subsection{How}
\begin{description}
\item[ Step 1.]觀察\\看到題目後先想想他有沒有一些特性
\item[ Step 2.]假設\\接者假設如果再做一個決策的時候,用怎麼樣的方法很有可能會朝向是最佳解的方向
\item[ Step 3.]證明\\證明每次的決策"這樣做不會更差"(結果會在最佳解上),然後增加coding信心
\end{description}


\subsection{Exercises!!}

\begin{description}
\item[ 1.]<賣場監視器>\\
給你每個人進出賣場的時間,問賣場什麼時候人最多?
\item[ 2.]<找零錢>\\
大多數國家的貨幣面額都設計過,使得用貪婪策略都剛好會是硬幣數量最少的最佳湊法?
\item[ 3.]<NPSC2005 pB 惱人的零錢>\\
咚咚想要買一個價格為C的物品,告訴你咚咚和老闆個別擁有的1,5,10,20,50硬幣的數量,請輸出找完錢後咚咚最少能剩下多少硬幣?
\item[ 4.]<TOI2015 二模>\\
有一些面額為1,5,10,50,100,500,1000的錢幣,問你用這些硬幣最小湊不出來的錢是多少?
\item[ 5.]<NPSC2005 pA 誰先早餐>\\
給你n個想要吃早餐的人和一個廚師。大家可以一起吃,但是廚師一次只能做一到菜。給定每人要吃多久,及他吃的那道菜要做多久,問至少需要多少時間所有人才能都吃完?
\item[ 6.]<Step5 0021 背包問題>\\
有兩個背包的容量分別為N,M現在要將N+M個物品放進背包,每件物品都有它的重量,求字典序最小的放法使兩個背包分別的重量平均數加起來最小。
\item[ 7.]<Step5 0031 蘿莉切割問題>\\
你要將一塊長L的木板切成N段,每段長度是Ai(A1+A2...+An=L),而將一塊長X的木板切開需要X的花費,問花費最小的方法?
\item[ 8.]<HOJ 28 看動畫>\\
你有一台有m個USB孔的電腦,還有編號1~N的隨身碟,現在你想要照一定的順序來看隨身碟裡的動畫,問最少的隨身碟插拔次數?
\item[ 9.]<TIOJ 1636 錢包的路>\\
地上有一排的錢包(每個相距1步的距離),你每走到一個錢包上面,就可以拿走裡面的錢。當你走一步路,地上錢包裡的錢就會恢復。問你走玩N步後,最多可以獲得多少錢?
\item[ 10.]<TIOJ 1432,1465 骨牌遊戲,H遊戲秘笈>\\
給定一個長為N的數列,求使分成K段之後每段總和的最大值最小的分法。
\end{description}


\end{document}